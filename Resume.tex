\documentclass[11pt,a4paper,roman]{moderncv}       



\moderncvstyle{banking}                            
\moderncvcolor{blue}                                
\nopagenumbers{}                                  


\usepackage[utf8]{inputenc}
\usepackage{fontawesome}
\usepackage{tabularx}
\usepackage{ragged2e}
\usepackage[scale=0.8]{geometry}
\usepackage{multicol}
\usepackage{import}

% personal data
\name{Mohammad}{Faraji}
\address{Software Engineer}{}{}
  
\newcommand*{\customcventry}[7][.25em]{
  \begin{tabular}{@{}l} 
    {\bfseries #4}
  \end{tabular}
  \hfill% move it to the right
  \begin{tabular}{l@{}}
     {\bfseries #5}
  \end{tabular} \\
  \begin{tabular}{@{}l} 
    {\itshape #3}
  \end{tabular}
  \hfill% move it to the right
  \begin{tabular}{l@{}}
     {\itshape #2}
  \end{tabular}
  \ifx&#7&%
  \else{\\%
    \begin{minipage}{\maincolumnwidth}%
      \small#7%
    \end{minipage}}\fi%
  \par\addvspace{#1}}

\newcommand*{\customcvproject}[4][.25em]{
%   \vfill\noindent
  \begin{tabular}{@{}l} 
    {\bfseries #2}
  \end{tabular}
  \hfill% move it to the right
  \begin{tabular}{l@{}}
     {\itshape #3}
  \end{tabular}
  \ifx&#4&%
  \else{\\%
    \begin{minipage}{\maincolumnwidth}%
      \small#4%
    \end{minipage}}\fi%
  \par\addvspace{#1}}

\setlength{\tabcolsep}{12pt}

%----------------------------------------------------------------------------------
%            content
%----------------------------------------------------------------------------------
\begin{document}
%-----       resume       ---------------------------------------------------------
\makecvtitle
\vspace*{-23mm}

\begin{center}
\begin{tabular}{ c c c c }
  \faEnvelopeO\enspace yasinfaraji100@gmail.com & \faMobile\enspace +989385545794 & \faMapMarker\enspace Tehran, Iran & \faCalendar\enspace 1999/12/06 \\  
\end{tabular}
\newline
Profiles:    
\href{https://www.linkedin.com/in/yasin-faraji-687567190/}{\faLinkedin}
 \href{https://github.com/YasinFaraji}{\faGithub} 
\end{center}


%----------------------------------------------------------------------------------------
%	SUMMARY SECTION
%----------------------------------------------------------------------------------------

\section{SUMMARY}

\cvitem{}{  Software Engineer,  C++ Developer,  Problem solving AND  +3 years of experience in C++ and Algorithms Mostly for system programming and desktop apps in Linux.}
\cvitem{}{  In my opinion, programming is an inseparable part of my life, because it takes up most of my daily life.
And when I am busy with this process and things like solving new problems, I don't notice the passage of time and this shows my passion and interest.
One of the attractions of this specialty is that in addition to becoming more skilled and gaining more experience every day, there are always more things that can be learned.
But in general, maintaining balance in work and life is very important. }


%----------------------------------------------------------------------------------------
%	EXPERIENCE SECTION
%----------------------------------------------------------------------------------------

\section{EXPERIENCE}
{\cventry{January  2021 -  January  2023 - 2 years}{Software Engineer}{Bontech Holding}{Tehran, Iran}{}
{\begin{itemize}
  \item  I have been working on the Telecommunications field. In addition to experience and general knowledge in this field, in particular, I have been responsible for the development, maintenance and improvement of important parts such as infrastructure cores and system distributors.
  In this company, I got acquainted with many technologies and learned many things in them; first of all in Linux, then in things like containers and Docker, also things like Kafka and message broker. In addition to this, I also found relatively good experience and knowledge in Redis, which was the result of working on a project related to having a database that stores, processes and retrieves high-speed data.
These were things minus the main work, i.e. programming. One of the important parts that added to my experience there was programming in Python effectively. So that 2 of the projects that I was responsible for developing and maintaining were with Python.
It was also the longest C++ project in this company. Important projects such as SMS center and messaging router and other such cases.
The experience of working in this environment is very informative because we always dealt with a large amount of data. And fast and simultaneous processing was much needed in these projects.
In general, the customers of this company were messaging companies and internet service providers. (such as Hamrah-e Avval, MTN Irancell, Shatel, etc.)
\item Main used technologies: C++, Python, Git, Linux, Threading, Concurrency, Docker, Kafka, redis, SQLite
\end{itemize}
}
}


{\cventry{May  2020 -  November  2020 - 7 months}{Software Engineer}{Sepehr Co}{Tehran, Iran}{}
{\begin{itemize}
  \item  I have been working on the military field and building desktop software and signal processing and telecommunications.
  The first place where I have experience working with C++ version 11 and higher was in this company.
Of course, one of the other advantages of this company was that it used the QT framework and it was very nice to work with this framework.
In general, they worked there on signal processing and statistical topics.
But one of the most interesting projects that I personally did there was that I designed and implemented a scheduler in which it was connected with the end user and had a relatively efficient user interface that, according to the user's input, in time It started recording or stopped recording a specified signal and also had the ability of long-term scheduling.
Besides this, there was also the implementation of mathematical algorithms.
\item Main used technologies: C++, Qt, MATLAB, Git, Threading, Concurrency
\end{itemize}
}
}

{\cventry{June 2019 - April 2020 - 1 year}{Developer}{Rayan Khodro Co}{Tehran, Iran}{}
{\begin{itemize}
  \item  I have been working on automotive diagnostics tools in one of the most reputable companies that make automotive diagnostics tools(Like a Diag device), which includes working on software and a little bit of hardware and network protocols.
  The most interesting part of working in that company was that a complete cycle of each project in that company included software, hardware and cars. and in this process, various challenges arose and it was very pleasant to finally see the output of the work.
Of course, this was the case that made our work not only programming in that company, and getting involved with hardware and things related to cars and even communicating with people such as mechanics and car experts were also features of this experience.
\item Main  Technologies used: C++, C Sharp, automotive diagnostics tools
\end{itemize}
}
}


%----------------------------------------------------------------------------------------
%	Voluntary-Activity SECTION
%----------------------------------------------------------------------------------------

\section{Voluntary activity (open sousrce community)}

{\cventry{March 2022 - Now }{Owner}{famOS (Operating System)}{Github}{}
{\begin{itemize}
  \item  This project is related to the implementation of an operating system.
  In this project, I tried to design and implement an operating system from the beginning and the base to a relatively complete state. Although this operating system is far from the current popular operating systems, it has been tried to be as similar as possible to an efficient and usable operating system.
  This operating system is in parts similar to Linux and in parts similar to Microsoft Windows.
  To run this operating system, you can use operating system simulators or boot it directly and run it on the hardware of a system in real form. Of course, this operating system does not have a graphical interface and is actually a server operating system.
  In this project, I tried to implement all the necessary items of an operating system, including the boot loader, which is one of the important parts of an operating system.
  Also, in other cases, for example, file system, etc., I tried to follow the standards.
  Finally, it should be noted that this current operating system is not yet officially usable and is only an early version of a relatively standard operating system.
  But this project was able to be completed and more efficient due to the type of design and implementation.
\item \href{https://github.com/YasinFaraji/famOS}{\textbf{github repository}}
\end{itemize}
}
}


{\cventry{September 2021 }{Contributor}{Gambal (Monitors network, CPU and memory usages)}{Github}{}
{\begin{itemize}
  \item  The purpose of this project is to monitor network values (for example, Internet consumption, etc.), CPU and RAM.
  This project is applicable only for Linux operating system (it is applicable in all Linux distributions).
  This tool has a very simple and understandable user interface that can be useful for programmers and even ordinary people during the day.
  The focus of this tool is to view the network (especially the download and upload amount of the Internet) and it is very light.
  I only collaborated in part of this project. In general, I added a feature to this tool that when you use this feature, all the current values will be reset, but the past history will remain unchanged. This feature is used to view and check the network momentarily.
\item \href{https://github.com/ashtum/gambal}{\textbf{github repository}}
\end{itemize}
}
}


%----------------------------------------------------------------------------------------
%	EDUCATION SECTION
%----------------------------------------------------------------------------------------

\section{EDUCATION}
\cventry{2018-2022}{Bachelor's degree in Computer Software Engineering}{Islamic Azad University Tehran}{Tehran, Iran}{}{- 18.11 GPA}

\cventry{2016-2018}{High School Diploma in Math and Physics}{Mandegar Alborz Highschool}{Tehran, Iran}{}{- 18.85 GPA}


%----------------------------------------------------------------------------------------
%	SKILLS SECTION
%----------------------------------------------------------------------------------------

\section{SKILLS}
\begin{minipage}{\maincolumnwidth}%
  \small{
      \begin{itemize}
	  \item Experienced  in \textbf{C++}, \textbf{C++11}, \textbf{C++14}, and Familiar with \textbf{C++17}, \textbf{C++20 (new features)}
          \item Experienced with \textbf{Linux}, \textbf{Ubuntu}, CentOS
          \item Experienced with \textbf{Git} and Gitlab, Github
	  \item Experienced with \textbf{Clean Code}
	  \item Familiar with \textbf{SOLID}, \textbf{Design Pattern}
    \item Familiar with \textbf{SQL} , MySQL , SQLite
    \item Familiar with \textbf{NoSQL} , Redis
	  \item Familiar with \textbf{Testing} and \textbf{Google Test(gtest)} and \textbf{TDD}
	  \item Familiar with \textbf{Python}
	  \item Familiar with \textbf{Docker}
	  \item Familiar with \textbf{kafka}
	  \item Familiar with \textbf{Qt}
    \item Familiar with \textbf{Assembly} for Kernel Development
	  \item Familiar with \textbf{MATLAB}
	  \item Familiar with \textbf{Unreal Engine}
          \item Good English communication skills
          \item Rapid Learner and eager to learn new technologies
    \end{itemize}}%
\end{minipage}%
      
% }

% }

\nocite{*}
\bibliographystyle{plain}
\bibliography{publications} 



\end{document}


